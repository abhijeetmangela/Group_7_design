\documentclass[12 pt]{article}
\usepackage[utf8]{inputenc}
\usepackage{graphicx}
\usepackage{amsmath}
\usepackage{amssymb}
\usepackage{multirow}
\usepackage{caption}
\usepackage{float}
\usepackage{subcaption}
\usepackage{hyperref}
\usepackage{pgf}
\usepackage{pgfpages}
\usepackage{textcomp}
\usepackage{lscape}
\usepackage{geometry}
\usepackage{pdflscape} 
\usepackage{placeins}
\usepackage{url}
\usepackage{natbib}
\usepackage[paper=A4]{typearea}
\graphicspath{ {figures/} }
\usepackage{array}
\usepackage{placeins}
\usepackage{afterpage}
\usepackage{bookmark}% faster updated bookmarks
%\usepackage[a4paper,margin=1in]{geometry}



\pgfpagesdeclarelayout{boxed}
{
  \edef\pgfpageoptionborder{0pt}
}
{
  \pgfpagesphysicalpageoptions
  {%
    logical pages=1,%
  }
  \pgfpageslogicalpageoptions{1}
  {
    border code=\pgfsetlinewidth{2pt}\pgfstroke,%
    border shrink=\pgfpageoptionborder,%
    resized width=.95\pgfphysicalwidth,%
    resized height=.95\pgfphysicalheight,%
    center=\pgfpoint{.5\pgfphysicalwidth}{.5\pgfphysicalheight}%
  }%
}

\pgfpagesuselayout{boxed}

\title{Assignment 1}
\author{Abhijeet Mangela}
\date{November 2022}

\title{Assignment 1}
\author{Abhijeet Mangela}
\date{\today}

\begin{document}
\begin{titlepage}
\begin{center}

\textbf{\huge Design project report Group 7 \\ \vspace{0.5 cm} Week 5} \\

\vspace{2 cm}

\centering
\includegraphics[width=0.4\textwidth]{IIT_Madras_Logo.svg.png}
\label{fig:my_label}

\vspace{2 cm}

\Large{Abhijeet Mangela AE21B040 \\ \vspace{0.2 cm} Navin Yadav AE23M803 \\ \vspace{0.2 cm} Balamurugan S AE23M009 \\ \vspace{0.2 cm} Samarth R Krishna AE23M032 \\ \vspace{0.2 cm} Senthil B AE23M035 \\ \vspace{0.2 cm} Rajendran Anandhu Nair AE23M027 }

\vspace{1.0 cm}

\textbf{\Large Department of Aerospace Engineering } \\ \vspace{0.2 cm}
\textbf{\Large IIT Madras} \\ \vspace{0.2 cm}
\textbf{\Large India} \\ 

\normalsize

\end{center}
\end{titlepage}

\newpage

\vspace{\fill}
\noindent

\begin{midpage}
    \centering
\textbf{Abstract} \\ \vspace{0.5 cm}

Air quality is an essential measure of the quality of life of any living being. A decrease in air quality is easily linked with a reduction in the life expectancy of various plants and animals.
As a result, we are focused on working on a drone that will give us an insight into the air quality of a region. 
We often have to measure air quality within a forest or a region where it is hard to go physically. As a result, we have to place expensive monitoring sensors at various challenging-to-reach locations. 
When some maintenance issues arise in these sensors, we again have to send teams to repair the equipment.
These complications can be reduced by using a fixed-wing UAV instead of ground sensors because the UAV is a moving object that can cover a larger area than a UAV sensor alone. Also, if some maintenance problem arises, it can be fixed when the UAV lands.
The UAV will also visually surveillance the land it is flying. A direct flight will provide a better and more frequent information intake than satellite imagery.
\end{midpage}

\vspace{\fill}


\newpage

\tableofcontents

\newpage

\thispagestyle{empty}
\listoffigures
\listoftables
\newpage

\textbf{\Huge{Chapter 1}}
\section{Introduction}

\subsection{Objective}
The main objective of this design project is to make a fixed-wing UAV that serves a multifaceted role in monitoring wildlife, detecting plastic pollution within designated zones, and assessing air quality, specifically targeting CO2, SO2, and other harmful gases, within amusement parks, zoos, and wildlife sanctuaries. Equipped with high-resolution cameras, infrared imaging technology, and advanced sensors, the UAV will conduct thorough wildlife surveys, identify plastic debris, and measure atmospheric conditions autonomously. By integrating these capabilities and employing data analytics techniques, the UAV will provide valuable insights for conservation efforts, environmental management, and visitor safety. This innovative solution underscores the potential of technology to address pressing environmental challenges while promoting sustainable practices in diverse ecosystems


\subsection{Mission Profile and requirements}

\subsubsection{Mission Requirement}
%Our objective is to detect the gases present near the airplane. It should detect elements including PM2.5, PM10, O3, NO2, SO2, CO, VOCs, H2S, NH3, HCl, CxHy, H2 and more.
In the context of crowd gatherings, preserving the cleanliness and ecological balance of natural environments poses a significant challenge. To address this issue effectively without dampening the morale of the crowd, unmanned aerial vehicles (UAVs) emerge as a pragmatic solution. By outfitting these drones with cutting-edge air quality sensors and high-resolution cameras, a comprehensive understanding of the terrain and environmental conditions can be attained. This technological integration allows for real-time monitoring of pollutants and potential hazards, such as plastics, while also facilitating detailed terrain mapping to identify sensitive ecosystems and wildlife habitats. Leveraging advanced data analysis techniques, the information collected by UAVs can be processed swiftly to formulate proactive action plans aimed at mitigating risks to flora and fauna. Moreover, drones serve as educational tools, engaging the crowd through captivating aerial footage to promote environmental awareness and stewardship. Seamlessly integrating with existing crowd management strategies, UAVs ensure that environmental protection remains a priority without compromising the safety or experience of event attendees. This scalable and adaptable approach underscores the potential of technology to harmonize crowd gatherings with ecological preservation, fostering a sustainable and responsible approach to communal celebrations and events. 

The UAV will be equipped with air quality sensors and cameras to understand the terrain better. Thus, knowledge about plastics and unpleasant atmospheres will help develop a quick action plan for maintaining the flora and fauna from foreign hazards.

\subsubsection{Aircraft Characteristics}
\begin{table}[h]
\centering
\resizebox{0.55\textwidth}{!}{%
\begin{tabular}{|c|c|}
\hline
\textbf{Estimated MTOW}            & 8 – 10 kg                \\ \hline
\textbf{Maximum Payload Weight}    & 1 – 1.5 kg               \\ \hline
\textbf{Estimated Endurance}       & 30 minutes               \\ \hline
\textbf{Mission ceiling}           & 0 - 250 m                \\ \hline
\textbf{Desired Operational Speed} & 17 m/s                   \\ \hline
\textbf{Transmitter Range}         & 4 km                     \\ \hline
\end{tabular}%
}
\caption{Initial Mission Requirements}
\label{Mission Requirements}
\end{table}

\subsubsection{Payload measure}
The payload of UAVs consists of a camera and a sensor module used to measure air quality. The camera will be a Hontral B0BMV12VYJ HD Camera weighing 492 g. The sensor module comprises Arduino MKR Proto Large Shield TSX00002, weighing about 200 g, and sensors such as MQ-7 CO Carbon Monoxide, MG811 Air Carbon Dioxide, etc., weighing about 320 g.

\subsection{Mission Profile}

\begin{figure}[h]
    \centering
    \includegraphics[width = \linewidth]{Drawing1-Model_final.pdf}
    \caption{Mission Profile}
    \label{Mission Profile}
\end{figure}


\subsubsection{Ground run}
The take-off ground run distance is approximately 60 meters, and the estimated take-off velocity is ten m/s. \cite{EgglestonUnknownTitle2015}

$$ (V_{TO})_{_{Bricans \: Td100}} = 19 \: m/s$$
$$ (W_{TO})_{_{Bricans \: Td100}} = 22.67 \: kg$$

Since $ V_{TO} \: \alpha \: \sqrt{W_{TO}} $ for given $C_L$ , S (applicable for initial estimate)

$$ (V_{TO})_{_{des}} = (V_{TO})_{_{Bricans \: Td100}} \times \sqrt{\frac{(W_{TO})_{_{des}}}{(W_{TO})_{_{Bricans \: TD \: 100}}}} $$

$$ = 19 \times \sqrt{\frac{5.6 \times 9.81}{22.67 \times 9.81}} = 9.94 \: m/s \approx 10 \: m/s $$

\subsubsection{Climb \cite{1000_questions} }
The UAV will climb at an estimated velocity of 17.6 m/s to an operating altitude of 250 m AMSL. The rate of climb is about 3.06 m/s. Time spent here: 81.7 s

Calculation :- 
$$ (V_{TO})_{_{des}} = 10 \: m/s \Rightarrow (V_{stall})_{_{des}} = \frac{(V_{TO})_{_{des}}}{1.2} = 8.33 \: m/s $$
$$ (V_{md})_{_{des}} = 1.6 \times V_{stall} = 13.33 \: m/s $$
$$ (V_{ROC})_{_{max}} = 1.32 \times V_{md} = 17.6 \: m/s $$

\subsubsection{Cruise \cite{alhajjaji2017design}}
The UAV will perform aerial surveillance and environmentally monitor for areas of plastics and solid waste with an estimated cruise speed of 17 m/s for the 20 km range. Time spent here is 1176 s.

\subsubsection{Descent to Mission Height}
The UAV will descend to an altitude of 150 m AMSL to study air quality at a sinking speed of 12.92 m/s for a 5 km range. Time spent here is 28 s.

Calculation: - 
$$ (V_{cr,lo})_{_{design}} = 0.76 \times (V_{cr})_{_{des}} \;  (initial \: estimate) $$
$$ = 12.92 \: m/s $$

\subsubsection{Descent to land \cite{Anderson1}}
The UAV will descend at an estimated velocity of 10.13 m/s. To close ground proximity, the UAV gradually decelerates to an estimated touchdown velocity of 8.61 m/s. Time spent here is 80 s.
Calculation: - 
$$(V_des)_{_{design}} = (V_{mg})_{_{design}} = (V_{mg})_{_{design}} \times 0.76 = 10.13 \: m/s $$

\subsubsection{Landing run}
The landing ground run distance is approximately 80 metres, and the estimated touchdown velocity is 8.61 m/s.

Calculation: - 
$$ (V_{TD})_{_{design}} = 0.85 \times (V_{des})_{_{design}} = 8.61 \: m/s  $$
$$ \text{Vertical component of } (V_{TD})_{_{design}} = 8.61 \times \sin{10^{\circ}} $$
$$ = 1.495 \: m/s \leq 4 \: m/s \; \text{(For smooth landing)} $$


\begin{figure}
    \begin{subfigure}{.4\textwidth}
        \centering
        \includegraphics[width = 0.9\linewidth]{Aircraft pics/WingtraOne.png}
        \caption{Wingtraone}
        \label{Wingtra one}
    \end{subfigure}
    \begin{subfigure}{.4\textwidth}
        \centering
        \includegraphics[width = 0.8\linewidth]{Aircraft pics/Albatross.jpg}
        \caption{Albatross}
        \label{Albatross}
    \end{subfigure}
    \begin{subfigure}{.4\textwidth}
    \centering
    \includegraphics[width=0.9\linewidth]{Aircraft pics/Azimut.jpg}
    \caption{Azimut 2}
    \label{Azimut 2}
    \end{subfigure}
    \begin{subfigure}{.4\textwidth}
        \centering
        \includegraphics[width = 0.9\linewidth]{Aircraft pics/Dragonfly.png}
        \caption{Dragonfly Tango 2}
        \label{Dragonfly Tango 2}
    \end{subfigure}
    \begin{subfigure}{.4\textwidth}
        \centering
        \includegraphics[width = 0.9\linewidth]{Aircraft pics/Nostroma.jpg}
        \caption{Nostroma Defensa Cadambra}
        \label{Nostromo Defence Cadambra}
    \end{subfigure}
    \begin{subfigure}{.4\textwidth}
        \centering
        \includegraphics[width = 0.9\linewidth]{Aircraft pics/spylite.png}
        \caption{Spylite}
        \label{Spylite}
    \end{subfigure}
    \centering
    \begin{subfigure}{.4\textwidth}
        \centering
        \includegraphics[width = 0.9\linewidth]{Aircraft pics/Skylark.jpg}
        \caption{Skylark}
        \label{Skylark}
    \end{subfigure}
    \caption{List of drones studied}
    \label{Drone pictures}
\end{figure}


\subsection{Data collection}

The data of all the parameters:-
% Please add the following required packages to your document preamble:
% \usepackage{graphicx}
\begin{table}[h]
\centering
\resizebox{\textwidth}{!}{%
\begin{tabular}{|c|c|c|c|c|c|c|c|}
\hline
SI no &
  UAV Name &
  \begin{tabular}[c]{@{}c@{}}MTOW \\ Kg\end{tabular} &
  \begin{tabular}[c]{@{}c@{}}Empty Weight\\ kg\end{tabular} &
  \begin{tabular}[c]{@{}c@{}}Battery Weight\\ kg\end{tabular} &
  \begin{tabular}[c]{@{}c@{}}Payload Weight\\ kg\end{tabular} &
  \begin{tabular}[c]{@{}c@{}}Range \\ km\end{tabular} &
  \begin{tabular}[c]{@{}c@{}}Endurance\\ min\end{tabular} \\ \hline
1 & Wingtraone   \cite{Wingtra}             & 4.5 & 2.387 & 0.604       & 1.509 &     & 59  \\ \hline
2 & Albatross    \cite{Albatross}             & 10  & 3.5   & 2.4         & 4.1   & 250 & 240 \\ \hline
3 & Azimut   2    \cite{Azimut}            & 9   & 2.5   & 2.8         & 3.7   &     &     \\ \hline
4 & Dragonfly   Tango 2  \cite{Dragonfly}      & 5   & 3     & 0.595+0.595 & 1.5   & 5   & 120 \\ \hline
5 & Nostromo   Defensa Cabure \cite{Nostromo} & 5   & 3.4   & 0.604       & 1     & 15  & 1.5 \\ \hline
6 & SPY   LITE    \cite{Bluebird}            & 9.5 & 4.5   & 1.5         & 1.35  & 80  & 240 \\ \hline
7 & Skylark I-LEX   \cite{Skylark}          & 7.5 & 5.5   & 0.8         & 1.2   & 40  & 180 \\ \hline
\end{tabular}%
}
\caption{Data part one}
\label{Data part one}
\end{table}

% Please add the following required packages to your document preamble:
% \usepackage{graphicx}
\begin{table}[h]
\centering
\resizebox{\textwidth}{!}{%
\begin{tabular}{|c|c|c|c|c|c|c|c|c|c|}
\hline
SI no &
  UAV Name &
  \begin{tabular}[c]{@{}c@{}}Service Ceiling\\ m\end{tabular} &
  \begin{tabular}[c]{@{}c@{}}Maximum Ceiling\\ m\end{tabular} &
  \begin{tabular}[c]{@{}c@{}}Cruise speed\\ km/hr\end{tabular} &
  \begin{tabular}[c]{@{}c@{}}Max Speed\\ km/hr\end{tabular} &
  \begin{tabular}[c]{@{}c@{}}Longitudinal Length\\ cm\end{tabular} &
  \begin{tabular}[c]{@{}c@{}}Wing Span\\ cm\end{tabular} &
  AR &
  L/D \\ \hline
1 & Wingtraone                & 120  & 2500   to 5000 & 35.8 &     &     & 125 & 1.838 &                \\ \hline
2 & Albatross                 &      &                & 68   & 129 &     & 300 &       & 28:1   to 30:1 \\ \hline
3 & Azimut   2                &      &                &      &     &     &     &       &                \\ \hline
4 & Dragonfly   Tango 2       &      &                & 43.2 & 100 &     &     &       &                \\ \hline
5 & Nostromo   Defensa Cabure & 4000 &                & 70   & 105 &     &     &       &                \\ \hline
6 & SPY   LITE                &      &                &      & 120 & 135 & 275 &       &                \\ \hline
7 & Skylark I-LEX             & 4572 &                & 37   & 74  & 220 & 300 &       &                \\ \hline
\end{tabular}%
}
\caption{Additional data}
\label{Data part 2}
\end{table}


\vspace{\fill}

%\Floatbarrier
%\vspace{\vfill}

\newpage

\afterpage{\clearpage}


\textbf{\Huge{Chapter 2}}

\section{Preliminary Weight estimation}

Weight estimation is divided into various sections
\subsection{Payload weight estimation}
Rough estimate for payload
\begin{table}[h]
\centering
\resizebox{0.6\textwidth}{!}{%
\begin{tabular}{|c|c|}
\hline
Purpose                            & Weight \\ \hline
Camera                             & 492 g  \\ \hline
Sensor module                      & 520 g  \\ \hline
Bulk Tolerance (Wiring, Actuation) & 388 g  \\ \hline
Total Payload Estimate             & 1400 g \\ \hline
\end{tabular}%
}
\caption{Payload weight estimate}
\label{Payload weight}
\end{table}

\hfill



\subsection{Empty weight estimation}
The empty weight is the weight of the UAV without the battery and the payload, essentially the structural weight of the UAV. 
This is the first weight calculation in the Design process. \\

\begin{figure}[h]
    \centering
    \includegraphics[width = 0.8\linewidth]{Codes/Week 2/Empty_weight.png}
    \caption{Empty weight estimation}
    \label{Empty Weight estimation}
\end{figure}


The empty weight ratio can be estimated from the previous data. 

We have fitted the data in a $y = A x^c$ curve with A and C as variables.

From our curve fit, we obtained A as 1.3887, and C as -0.5155

Using the above equation, the empty weight of our UAV is 3.1746 kg.

\newpage

\subsection{Battery weight estimation}

The battery is an essential criterion for the design of any drone. We should optimize the battery weight using the Mission profile we have. That way, we will not be carrying unnecessary dead weight with us.

For calculating Battery weight estimation, we must first find out the total energy required for each Major phase of flight.

\subsubsection{Cruise}
Now, for a level flight,
$$ L = W \; \; , \; \; T = D$$

$$ W = \frac{1}{2} \rho V^2 S C_L $$
$$ C_L = \frac{W}{\frac{1}{2} \rho V^2 S}$$
So,
$$ C_D = C_{D_0} + \frac{C_L^2}{\pi e AR} $$
Now 
$$ P = T\times V = D\times V $$

So, 
$$ P_R = \frac{1}{2}\rho V^3 S C_{D_o} + \frac{2 W^2}{ (\rho V S) \pi e AR} $$

$$P_R = f(W,\rho,V,C_{D_o},S,e,AR)$$

Taking approximations and inputting mission profile conditions

$$ P_R = f(W) $$

We will consider the Loitor phase as a cruise, too.

For our cruise phase of 17 m/s, we will need a power of 84.35 W.

\subsubsection{Climb }
The power required for climb is given by 
$$ P = W \times R.O.C + P_R $$

Climbing is also the portion where we need a good amount of power. For our estimated weight, we need a preliminary power of 251 W.

\subsubsection{Descent}
The power required for descent is 
$$ P = P_R - W \times Rate\: of\: descent$$
We need to ensure the power does not go negative here or make it zero while estimating energy. The extra allowance we give will take care of the power consumed.

\subsubsection{Take off and Landing}
The Segments of take-off and Landing do not take that much time. Also, estimating the energy spent in takeoff and Landing is harder due to the carrying power needs. So, we will take an excess power of 20 percent to account for the errors in different power situations.

\subsubsection{Battery type selection \cite{Lipobattery}}
We will use a lithium polymer battery for our aircraft. Some of the advantages offered by a lithium polymer battery are that it is less likely to leak away and still rivals Lithium-ion batteries in terms of energy density. It is bulkier by volume, but by weight, it is comparable.

\begin{figure}[h]
    \centering
    \includegraphics[width=0.5\linewidth]{Extra pics/LiPo_battery_diagram.png}
    \caption{Lithium Polymer battery}
    \label{Lithium Polymer battery}
\end{figure}

Lithium Polymer batteries are rechargeable batteries that transmit electricity due to the difference in charge between the Cathode and Anode parts of the battery. While pretty safe to handle, care should be taken not to get the battery near any fore source, or else it tends to catch fire.

We can estimate the energy in a battery by using this formula \cite{Lipobattery}:-

$$ \text{Battery Capacity} = \sigma \times \text{M}_\text{battery} $$

We used a commercially available battery to calculate the battery energy density of it \cite{Batteryref}. The energy density came out to be 237600 J/kg.

\subsubsection{Battery weight}

We can calculate the energy by multiplying the power required by the time spent in that phase. From energy, we can get battery weight by dividing it by energy density. 

In our case, the battery weight came out to be 0.933 kg.

%For lithium polymer batteries, the energy density is around 273600.

\subsection{Total weight estimation}

The total weight is given as:- 
$$ W_0 = W_{pl} + W_{e} + W_{f} $$
Where,
\begin{itemize}
    \item[-] $W_0$ is the total design weight.
    \item [-] $W_{pl}$ is the payload weight.
    \item [-] $W_{e}$ is the empty weight.
    \item [-] $W_{f}$ is the fuel weight (Battery in our case).
\end{itemize}

It can be transformed into:- 
$$W_{0} = \frac{W_{pl}}{1 - \frac{W_{e}}{W_{0}} - \frac{W_{f}}{W_0}}$$

If we have an initial approximation, we can run an iteration algorithm to determine the design weight when we keep the empty weight ratio as a variable.

%Generally, we take an initial estimate to be four times the payload weight. Taking the correct initial weight is important,, or the code may blow up.

In our case, we use a special algorithm to preserve the momentum of the descent from the previous iteration. This is done because otherwise, the code will blow up and reach a negative value. 

$$ W_o = W_{\text{Previous iteration}} + W_{\text{New iteration}} $$

The code for the algorithm is mentioned below in the GitHub reference.

In our case, the design weight was 6.6102 kg after taking a 20 percent leverage over the original weight.

\begin{figure}
    \centering
    \includegraphics[width=0.75\linewidth]{Codes/Week 2/weight.png}
    \caption{Weight Estimation}
    \label{Weight Estimation}
\end{figure}

\vfill

\afterpage{\clearpage}

\newpage

\textbf{\Huge{Chapter 3}}

\section{Calculation of Aerodynamic Coefficients}

\subsection{\ Calculation of Aspect Ratio and Lift Coefficient}

We first start calculating the wing reference area and wetted area of all our reference UAVs and then proceeded to calculate the reference aspect ratio as well as the wetted aspect ratio of all the UAVs. We have used ImageJ and Fusion 360 software for estimating the Reference area and wetted area of all the UAV.\\

\textbf{Wingspan} : It is the distance between the wingtips of the UAV.\\

\textbf{Mean aerodynamic chord} : It is the average of the chord lengths across the cross sections of the wings.\\

\textbf{Aspect Ratio} : The aspect ratio is crucial as it directly influences its aerodynamic performance and efficiency. Higher aspect ratios generally result in lower induced drag and better lift-to-drag ratios, enabling improved endurance and range, which are essential factors in UAV design optimization.

The aspect ratio ($AR$) of a rectangular wing aircraft is calculated using the formula:
\[
AR = \frac{{\text{Wingspan}}}{{\text{Mean aerodynamic chord}}}\]
where wingspan is the distance between the wingtips and the Mean aerodynamic chord is the average chord length of the wing.\\
By using steady-level conditions, where Lift = Weight, we have calculated the Lift Coefficient at cruise conditions.\\
When the lift force ($L$) of an aircraft equals its weight ($W$), the coefficient of lift ($C_L$) can be calculated using the formula:

\[
C_{L_{cruise}} = \frac{W}{\frac{1}{2} \times \rho \times V^2 \times S_{\text{ref}}}
\]
where $\rho$ is the density of air, $V$ is the velocity of the aircraft, and $S_{\text{ref}}$ is the reference wing area.
\subsubsection{Sample calculation of Aspect ratio and Lift Coefficient for cruise condition of Wingtraone UAV}
From the data collection done in week 2, we have estimated cruise velocity for our UAV as $V_{cruise}$ = 16m/s
\begin{itemize}
  \item Wingspan ($b$) = 125 cm (obtained from data collection)
  \item Reference wing area ($S_{\text{ref}}$) = 5065 cm\textsuperscript{2} (estimated from image processing)
\end{itemize} 

To calculate the aspect ratio (AR) using the formula $AR = \frac{b^2}{S_{\text{ref}}}$:

\[
AR = \frac{125^2}{5065} = \frac{15625}{5065} \ =  3.08
\]

So, the aspect ratio is 3.085.\\

Given:
\begin{align*}
    \text{Maximum Takeoff Weight (MTOW)} & = 44.15 \, \text{N} \\
    \rho & = 1.225 \, \text{kg/m}^3 \\
    S_{\text{ref}} & = 0.507 \, \text{m}^2 \\
    v_{\text{cruise}} & = 16 \, \text{m/s}
\end{align*}


For cruise condition, we know that Lift = Weight.

{Calculating Coefficient of Lift for Cruise Conditions}

Estimation of $(L/D)_{\text{max}}$:

For steady cruise conditions, we know:

\begin{align}
\left(\frac{T}{W}\right)_{\min} &= \frac{1}{(L/D)_{\text{max}}} \tag{3.3} \\
\left(\frac{L}{D}\right)_{\text{max}} &= \sqrt{\frac{1}{\ 4CD_0 k}} = \sqrt{\frac{\pi e AR}{\ 4CD_0 k}} \tag{3.4}
\end{align}

Equation 3.3 has been used to determine $(L/D)_{\text{max}}$ for each of the aircraft. From Equation 3.3, it is evident that we need to determine $CD_0$ in order to obtain $(L/D)_{\text{max}}$.

Since it is Steady Cruise, the Equations of motion and aerodynamic relations are as follows:

\begin{align}
L &= W_0g \tag{3.5} \\
C_L &= \frac{2W_0g}{\rho V^2_{\text{cr}} c_r S_{\text{ref}}} \tag{3.6} \\
C_D &= CD_0 + kC_L^2 \tag{3.7}
\end{align}

Equation 3.6 is used to determine $C_L$ for each of the aircraft. Input parameters such as $V_{\text{cr}}$ (Velocity in cruise) and $S_{\text{ref}}$ vary for each aircraft. $V_{\text{cr}}$ is obtained for each aircraft during the data collection phase and $S_{\text{ref}}$ is determined following the method stated in the above section.

For $(L/D)_{\text{max}}$:

$(L/D)_{\text{max}}$ and $AR_{\text{wet}}$ for each aircraft are obtained using Equations 3.4 and 3.3 respectively. The data obtained during this exercise is tabulated in Table 5.

Substituting the given values:
\[
C_{D0} = 0.1475 \times (0.556)^2
\]

Calculating \( C_{D0} \):
\[
C_{D0} \ = 0.1475 \times (0.556)^2 \ = 0.0461
\]

Therefore, the minimum drag coefficient is approximately \(0.0461\).\\
Now Let's calculate \( L/D_{\text{max}} \))
\begin{align*}
    k & = 0.1475 \\
    C_{D0} & = 0.046
\end{align*}

Equating lift to weight for cruise conditions:
\[
L = W
\]
\[
\frac{1}{2} \rho v_{\text{cruise}}^2 S_{\text{ref}} C_{L_{\text{cruise}}} = W_{\text{cruise}}
\]

Substituting the given values:
\[
\frac{1}{2} \times 1.225 \times 16^2 \times 0.507 \times C_{L_{\text{cruise}}} = 44.15
\]

Solving for $C_{L_{\text{cruise}}}$:
\[
C_{L_{\text{cruise}}} = \frac{44.15}{\frac{1}{2} \times 1.225 \times 16^2 \times 0.507} = 0.556
\]

For maximum lift-to-drag ratio (\( L/D_{\text{max}} \)):
\[
\frac{L}{D_{\text{max}}} = \frac{1}{2\sqrt{k \cdot C_{D0}}}
\]

Substituting the given values:
\[
\frac{L}{D_{\text{max}}} = \frac{1}{2\sqrt{0.1475 \cdot 0.046}}
\]

Calculating \( L/D_{\text{max}} \):
\[
\frac{L}{D_{\text{max}}} \ =  \frac{1}{2\sqrt{0.1475 \cdot 0.046}} = 6.103
\]

For minimum drag lift-to-drag ratio (\( L/D_{\text{minimum drag}} \)), it's the same as \( L/D_{\text{max}} \) due to the assumption that \( L/D_{\text{max}} = L/D_{\text{minimum drag}} \).

Therefore, both \( L/D_{\text{max}} \) and \( L/D_{\text{minimum drag}} \) are approximately \(6.103\).

\begin{table}[h]
\centering
\resizebox{\textwidth}{!}{%
\begin{tabular}{|c|c|c|c|c|c|c|c|}
\hline
SI no &
  UAV Name &
  \begin{tabular}[c]{@{}c@{}}MTOW \\ Kg\end{tabular} &
  \begin{tabular}[c]{@{}c@{}}Cruise speed\\ m/s\end{tabular} &
  \begin{tabular}[c]{@{}c@{}}$AR_{ref}$\end{tabular} &
  \begin{tabular}[c]{@{}c@{}}Wetted AR\end{tabular} &
  \begin{tabular}[c]{@{}c@{}}$C_{L}{cruise}$\end{tabular} &
  \begin{tabular}[c]{@{}c@{}}$\text{L/D}_{max}$\end{tabular} \\  \hline
1 & Wingtraone                & 4.5 & 16 & 3.085       & 1.874 &   0.556  & 6.103  \\ \hline
2 & Albatross                & 10  & 18.89   & 11.688         & 5.939   & 0.583 & 22.047 \\ \hline
3 & Azimut   2    & 9   & 13.89   & 5.776         & 2.992   & 0.690    &  9.198   \\ \hline
4 & Birdeye  600        & 8.5   & 20.57    & 8.948 & 4.474   & 0.452   & 21.772 \\ \hline
5 & Rafael Skylite BR & 8   & 30.56   & 9.969       & 5.145     & 0.608  & 18.024 \\ \hline
6 & Bluebird  Boomerang             & 9.5 & 22.12   & 5.948         & 3.118  & 0.640  & 10.219 \\ \hline
\end{tabular}
}
\caption{Data part two}
\label{Data part two}
\end{table}

\begin{figure}
    \centering
    \includegraphics[width=0.8\linewidth]{graph.png}
    \caption{$\text{(L/D)}_{max}$ \text{v/s} $\sqrt{AR_{wet}}$ }
    \label{fig:enter-label}
\end{figure}

\newpage

\section{Power Loading for Different mission segments}
Given:
\begin{align*}
& L/D_{\text{max}} = 14.56067 \quad (\text{Taken from average of } L/D_{\text{max}} \text{ of data collection}) \\
& L/D_{\text{max}} = 30.249 \ln(\sqrt{AR_{\text{wet}}}) - 5.0483 \\
& \sqrt{AR_{\text{wet}}} = 1.912198 \\
& AR_{\text{wet}} = 3.6565 \\
& \frac{S_{\text{wet}}}{S_{\text{ref}}} = 1.9626 \quad (\text{From Data Collection and plotting graph between} \\
& \quad \text{ } \frac{S_{\text{wet}}}{S_{\text{ref}}} \text{ and } \sqrt{AR_{\text{wet}}}) \\
& AR_{\text{ref}} = \frac{b^2}{S_{\text{ref}}} = \left(\frac{b^2}{S_{\text{wet}}}\right) \left(\frac{S_{\text{wet}}}{S_{\text{ref}}}\right) = AR_{\text{wet}} \left(\frac{S_{\text{wet}}}{S_{\text{ref}}}\right) = 3.6565 \times 1.9626 = 7.175 \\
& e = 0.7 \quad (\text{for rectangular wing taken from NASA website}) \\
& k = \frac{1}{\pi e AR_{\text{ref}}} = \frac{1}{\pi \times 0.7 \times 7.175} = 0.063377 \\
& S_{\text{ref}} = 0.7 \text{ m}^2 \quad (\text{design requirement}) \\
& b = \sqrt{AR_{\text{ref}} S_{\text{ref}}} = 2.241098 \text{ m} \\
& c = \frac{b}{S_{\text{ref}}} = 0.312348 \text{ m} \\
& L/D_{\text{max}} = \frac{1}{2\sqrt{k C_{d0}}} \\
& C_{d0} = \frac{1}{4} \left(\frac{L}{D}\right)^2 k \\
& C_{d0} = 0.018606 \\
& \text{For NACA 4415 Airfoil:} \\
& C_{L_{\text{max, 2D}}} = 1.3393 \quad (\text{from airfoiltools.com}) \\
& C_{L_{\text{max, 3D}}} = 0.9 C_{L_{\text{max, 2D}}} = 1.2054 \\
& V_{\text{stall}} = 11.2 \text{ m/s} \quad (\text{design requirement}) \\
& V_{\text{stall}} = \sqrt{\frac{2(W/S)}{\rho C_{L_{\text{max, 3D}}}}} \\
& \frac{W}{S} = \frac{\rho V_{\text{stall}}^2 C_{L_{\text{max, 3D}}}}{2} \\
& \frac{W}{S} = 92.65865 \text{ N/m}^2 \\
\end{align*}
\subsection {Power Loading for Climb}
\begin{align*}
& V_{\text{climb}} = V_{\text{minimum power}} = \sqrt{\frac{2W}{\rho S}} \left(\frac{k}{3C_{d0}}\right)^{\frac{1}{4}} \\
& V_{\text{climb}} = 12.69456 \text{ m/s} \\
& V_{\text{takeoff}} = 1.2 V_{\text{stall}} = 1.2 \times 11.2 = 13.44 \text{ m/s} \\
& S_{\text{rotation}} = V_{\text{rotation}} t_{\text{rotation}} = 13.44 \times 1.5 = 20.16 \text{ m} \quad (\text{ } t_{\text{rotation}} = 1.5 \text{ seconds for small airplanes - From JD Anderson}) \\
& R_{\text{rotation}} = 6.96 \times \frac{V_{\text{takeoff}}^2}{9.81} = 68.99719 \text{ m} \\
& \text{Climb angle} = \theta_{\text{c}} = \sin^{-1}\left(\frac{S_{\text{rotation}}}{R_{\text{rotation}}}\right) = 13.0995^\circ \\
& \text{ROC}_{\text{max}} = V_{\text{climb\_ROCmax}} \sin(\theta_{\text{c}}) = 12.6945 \sin(13.09915^\circ) = 2.8770757 \text{ m/s} \\
& \eta_{\text{prop}} \frac{P}{W} = \text{ROC}_{\text{max}} + \sqrt{\left(\frac{2(W/S) (k/3C_{d0})^{\frac{1}{2}}}{\rho}\right)^{\frac{1}{2}} \left(\frac{1.155}{(L/D)_{\text{max}}}\right)} \quad (\text{From JD ANDERSON}) \\
& \eta_{\text{prop}} \frac{P}{W} = 3.014397 \\
& \eta_{\text{prop}} = 0.8 \quad (\text{From JD Anderson}) \\
& \left(\frac{P}{W}\right)_{\text{climb}} = \frac{3.014397}{0.8} = 3.767996
\end{align*}
\subsection {Power Loading for Cruise}

\begin{center}
$V_{\text{cruise}} = 17 \, \text{m/s}$
\end{center}
\begin{align*}
\frac{T}{W}_{\text{cruise}} &= \frac{1}{(L/D)_{\text{cruise}}} = 0.068678 \\
\frac{P}{W}_{\text{cruise}} &= \frac{T}{W}_{\text{cruise}} \cdot V_{\text{cruise}} = 1.167529 \\
\frac{P}{W}_{\text{cruise}} &= 1.167529 \\
\end{align*}

\subsection {Power Loading for Descent}

\begin{center}
$V_{\text{approach}} = 14.56 \, \text{m/s}$ \\
$V_{\text{flare}} = 13.776 \, \text{m/s}$
\end{center}
\begin{align*}
\theta_{\text{descent}} &= \sin^{-1}\left(\frac{S_{\text{flare}}}{R_{\text{flare}}}\right) = 11.52162^\circ \\
V_{\text{cruise to descent}} &= V_{\text{cruise}} \cdot \cos(\theta_{\text{descent}}) = 16.65 \, \text{m/s} \\
V_{\text{average}} &= \frac{V_{\text{cruise to descent}} + V_{\text{approach}}}{2} = 15.605 \, \text{m/s} \\
D_{\text{descent}} &= D_{\text{cruise}} \cdot \cos(\theta_{\text{descent}}) = 4.520132 \, \text{N} \\
P_{\text{descent}} &= D_{\text{descent}} \cdot V_{\text{average}} = 70.53665 \, \text{W} \\
\end{align*}
\begin{center}
$\frac{P}{W}_{\text{descent}} = 1.087755$
\end{center}
\subsection {Power Required for Climb}

\begin{align*}
\frac{P}{W}_{\text{climb}} &= 3.767996 \\
P_{\text{climb}} &= \frac{P}{W}_{\text{climb}} \cdot W \\
&= 3.767996 \times 64.84 \\
&= 244.63 \, \text{W}
\end{align*}

\subsection {Power Required for Cruise}

\begin{align*}
\frac{P}{W}_{\text{cruise}} &= 1.167529 \\
P_{\text{cruise}} &= \frac{P}{W}_{\text{cruise}} \cdot W \\
&= 1.167529 \times 64.84 \\
&= 75.70 \, \text{W}
\end{align*}

\subsection {Power Required for Descent}

\begin{align*}
\frac{P}{W}_{\text{descent}} &= 1.087755 \\
P_{\text{descent}} &= \frac{P}{W}_{\text{descent}} \cdot W \\
&= 1.087755 \times 64.84 \\
&= 70.52 \, \text{W}
\end{align*}

\subsection {Power Required for Loiter}

Since we have assumed that loitering would be done in cruise conditions, we will take all cruise condition values and calculate the power required for loitering.

\begin{align*}
\frac{P}{W}_{\text{loiter}} &= 1.167529 \\
P_{\text{loiter}} &= \frac{P}{W}_{\text{loiter}} \cdot W \\
&= 1.167529 \times 64.84 \\
&= 75.70 \, \text{W}
\end{align*}

%The total power required is the sum of the powers required for climb, cruise, descent, and loiter:

%\[
%\text{Total Power Required} = P_{\text{climb}} + P_{\text{cruise}} + %P_{\text{descent}} + P_{\text{loiter}}
%\]
%\[
%= 244.63 + 75.70 + 70.52 + 75.70 = 466.55 \, \text{W}
%\]

%Given that the power required for takeoff and Landing is negligible, we'll add a tolerance of 20\% to the total power value:

%\[
%\text{Final Total Power Required} = 466.55 + (0.20 \times 466.55) = 466.55 + %93.31 = 559.86 \, \text{W}
%\]
%\textbf{The total power required for the UAV is $559.86 \, \text{W}$.}



\subsection {UAV powerplant selection}

\subsubsection {Battery Capacity Calculation}

For each mission segment, we will calculate the energy required in watt-hours by multiplying the power needed for that segment by its endurance time in hours.

\begin{itemize}
    \item Endurance for climb = 2 minutes = 0.0333 hours
    \item Endurance for cruise = 12 minutes = 0.2 hours
    \item Endurance for loiter = 10 minutes = 0.1667 hours
    \item Endurance for descent = 2 minutes = 0.0333 hours
    \item Endurance for takeoff and landing = 4 minutes = 0.0667 hours
\end{itemize}

\begin{align*}
\text{Energy for climb} &= P_{\text{climb}} \times \text{Endurance for climb} \\
&= 244.63 \times 0.0333 \\
&= 8.15 \, \text{Wh} \\
\\
\text{Energy for cruise} &= P_{\text{cruise}} \times \text{Endurance for cruise} \\
&= 75.70 \times 0.2 \\
&= 15.14 \, \text{Wh} \\
\\
\text{Energy for loiter} &= P_{\text{loiter}} \times \text{Endurance for loiter} \\
&= 75.70 \times 0.1667 \\
&= 12.61 \, \text{Wh} \\
\\
\text{Energy for descent} &= P_{\text{descent}} \times \text{Endurance for descent} \\
&= 70.52 \times 0.0333 \\
&= 2.35 \, \text{Wh} \\
\\
\text{Energy for takeoff and landing} &= (\text{Final Total Power Required} \times 0.20) \times \text{Endurance for takeoff and landing} \\
&= (559.86 \times 0.20) \times 0.0667 \   \text{we take 20 percent buffer of total power}  \\
&= 7.47 \, \text{Wh} \\
\end{align*}

The total energy required is the sum of energies for all mission segments:

\[
\text{Total Energy Required} = 8.15 + 15.14 + 12.61 + 2.35 + 44.45 = 82.7 \, \text{Wh}
\]

To calculate battery capacity, we'll use the formula:

\[
\text{Capacity (mAh)} = \frac{\text{Energy Required (Wh)}}{\text{Voltage (V)}} \times 1000
\]


Now, let's assume a typical lithium-ion battery voltage of 14.8V for calculation purposes.

\begin{align}
\text{Battery Capacity} &= \frac{45.72}{14.8} \times 1000 \\
& = 3089.18 \, \text{mAh}
\end{align}

%Bro give powerplant weight - Propeller and motor weight
%Write the answer here
%

So, the total battery capacity required for the aircraft is approximately $3089.18 \, \text{mAh}$.

\subsubsection{Battery and Motor Selection}

\subsubsection{Battery Selection}

To meet the power requirements with a tolerance of about 1000 mAh, we have selected a MaxAmps Lithium Polymer (LiPo) battery with a capacity of 4000mAh operating at a voltage of 14.8V.

\begin{figure}[h]
    \centering
    \includegraphics[width=0.7\textwidth]{LiPo-4000-4S-14.8v-Battery-Pack.jpg}
    \caption{LiPo Battery}
    \label{fig:battery}
\end{figure}

\begin{table}[h]
    \centering
    \caption{MaxAmps Lithium Battery Specifications}
    \begin{tabular}{|l|l|}
    \hline
    \textbf{Specification} & \textbf{Value} \\ \hline
    Brand & MaxAmps Lithium Batteries \\
    Capacity & 4000mAh \\
    Maximum Voltage & 16.8V \\
    Minimum Voltage & 12V \\
    Recommended Landing Voltage (Air) & 14V \\
    Recommended Cut-off Voltage (Ground) & 12.8V \\
    Chemistry & Lithium-Polymer (LiPo) \\
    Maximum Continuous Discharge & 128A \\
    Maximum Charge Current & 20A \\
    Watt Hours & 59.2Wh \\
    Energy Density & 151 Wh/kg \\
    Main Lead Length (Custom lengths available) & 5.5" (140mm) \\
    Balance Lead Length (Custom lengths available) & 5.5" (140mm) \\
    Length & 5.39" (138mm) \\
    Width & 1.85" (45mm) \\
    Height & 1.14" (48mm) \\
    Weight & 388g \\ \hline
    \end{tabular}
\end{table}

\subsubsection{Motor and Propeller selection}

We have selected AT 2317 Long Shaft KV 1440 Motor with APC 8x6 propeller to meet the above calculated power requirements and voltage capacity. All the specifications are mentioned below.

Selecting the right propeller is crucial for efficient performance and optimal thrust generation. We have selected an APC 8x6 propeller that is compatible with our motor.

\begin{figure}[h]
    \centering
    \includegraphics[width=0.3\textwidth]{motorr.jpg}
    \caption{AT2317 LONG SHAFT KV 1440 MOTOR}
    \label{fig:motor}
\end{figure}

\begin{figure}[h]
    \centering
    \includegraphics[width=0.8\textwidth]{motor specifications.png}
    \caption{MOTOR SPECIFICATIONS \cite{enginebuy}}
    \label{fig:motor_specifications}
\end{figure}

\begin{figure}[h]
    \centering
    \includegraphics[width=0.5\textwidth]{propeller.jpg}
    \caption{APC 8x6 PROPELLER }
    \label{fig:propeller}
\end{figure}

\begin{figure}[h]
    \centering
    \includegraphics[width=0.6\textwidth]{propeller specifications.png}
    \caption{PROPELLER SPECIFICATIONS}
    \label{fig:propeller_specifications}
\end{figure}

\newpage

\clearpage
\begin{table}[h]
    \centering
    \caption{Powerplant Components and Weights}
    \label{tab:powerplant_weights}
    \begin{tabular}{|l|c|}
        \hline
        \textbf{Powerplant Component} & \textbf{Weight (g)} \\
        \hline
        Battery & 388 \\
        Motor & 80 \\
        Propeller & 19 \\
        \hline
    \end{tabular}
\end{table}

\newpage

\textbf{\Huge{Chapter 5}}

\section{Wing Loading}

\subsection{Stall Criteria}
According to \cite{stall1} Stall speed is the minimum speed at which the airplane remains controllable during it's flight for a steady cruise. This is the point where the $C_L$ of the plane is maximum i.e. $C_{L_{\text{max}}}$

At this speed the Angle of attack of the aircraft becomes so large that the flow begins to separate at the top of the airfoil resulting in a drop of lift if we increase the angle of attack beyond that.

\begin{figure}[h]
    \centering
    \includegraphics[width=0.4\linewidth]{Extra pics/Cllvsalpha.png}
    \caption{Variation of $C_L$ with angle of attack ref- \cite{stallpic}}
    \label{Variation of $C_L$ with angle}
\end{figure}

\subsection{ $\frac{W}{S}$ calculations }

\subsubsection{Stall}

According to our take-off speed based on the Mission profile which is 10 m/s we will take our stall speed to be 8.33 m/s

At stall we know,
$$W = \frac{1}{2} \rho V^2 S C_{L_{max}} $$

So, $\frac{W}{S}$ comes out to be 107.5648 $N/m^2$. So, taking our estimated weight, the S comes out to be 1.0879 m.

%Since we are not dropping any payload or using any fuel, our $\frac{W}{S}$ will remain the same overall. But in the case of turning, one wing can have a higher loading than the other to balance out forces. But the overall $\frac{W}{S}$ will remain the same.

\subsubsection{Take off parameter}

According to \cite{Raymer.2006}, Raymer states that the Takeoff Parameter is given as:- 

$$\text{TOP} = \frac{\frac{W}{S}}{ \sigma C_{L_{TO}} \left( \frac{P}{W} \right) }$$
where,
\begin{itemize}
    \item[-] $\sigma$ is the density ratio we will take as 1.
    \item [-] $C_{L_{TO}}$ will be the Coefficient of lift at Takeoff 
\end{itemize}

We have taken the takeoff distance to be 80 m, which is around 200 feet. Changing the previous equation:-

$$ \frac{W}{S} = \text{TOP} \times \sigma C_{L_{TO}} \left( \frac{P}{W} \right) $$

This equation will estimate the maximum $\frac{W}{S}$ which we can take for our aircraft.

The TOP is given in ref \cite{Raymer.2006} page number 130 in the form of a graph. The graph was digitalized using Image processing, and a Linear regression was taken in Figure:-\ref{Image processed Take off parameter}. 

We will take a power of 500 W for take-off, which can be provided by out power-plant at 80 \% throttle. 

\begin{figure}[h]
    \centering
    \includegraphics[width=0.8\linewidth]{Codes/Week 2/Takeoffparam.png}
    \caption{Image processed Take off parameter from \protect\cite{Raymer.2006} pg no 130}
    \label{Image processed Take off parameter}
\end{figure}


For our case, we calculated that the maximum Wing loading during take-off can be 352 $N/m^2$. This is sufficiently higher than the wing loading by Stall criteria, so we can rest assured that our plane can take off properly.

\subsubsection{Landing distance}

According to \cite{Raymer.2006} Chapter 5.3.5, Raymer states that Landing distance is the minimum distance the plane will take to come to a halt.

It includes a clearing distance as well as a ground run distance. The landing distance heavily relies on $C_{L_{max}}$, which can be increased heavily by using flaps.

It is given as:- 
$$ S_{\text{Landing}} = 5*\left( \frac{W}{S} \right) \left( \frac{1}{\sigma C_{L_{\text{max}}}} \right) \: + \: S_a$$
where,
\begin{itemize}
    \item [-] $\sigma$ is the density ratio we will take as 1.
    \item[-] $C_{L_{\text{max}}} $ is the maximum Lift coefficient which we will take as 2 with flaps.
    \item[-] $S_a$ is the 'Clearance Distance' we should take before the actual ground roll to stay clear of all objects. We will take it as 50 m.
\end{itemize}

Using this, the Current landing estimate comes out to be 198.8 m, along with the Clearance distance. It should be noted that the distance can be significantly shortened by using flaps more efficiently to increase the maximum lift coefficient.

Also, some methods, like thrust reversion, can be considered in the design process to reduce landing distance.


\subsubsection{Cruise}

Raymer in \cite{Raymer.2006} ch 5.3.7 discusses that for maximising range in cruise, we would like a Wing loading, which is generally much higher than what is required for a wing required for stall speed. As a result, making a wing based on this estimate is pretty unsafe and will drastically increase stall speed.

The propeller aircraft will give us the maximum range when we have the highest L/D ratio. It is speed when the parasitic drag equals Induced drag as shown in Chapter 17 of \cite{Raymer.2006}. Thus:-
$$qSC_{D_o} = qS\frac{C_L^2}{\pi e AR}$$
It will transform to give us:- 
$$ C_L = \sqrt{C_{D_o} \pi e AR }$$

This can be substituted back into the Cruise equation for the lift to get:-
$$ \frac{W}{S} = q \sqrt{C_{D_o} \pi e AR}$$
where, q is $\frac{1}{2} \rho V^2 $

For our cruise condition, our Wing loading comes out to be 95.91 $N/m^2$. This is expected for a perfect cruise, so it will be best to have as small a wing as possible to minimise impossible drag.

\subsubsection{Loiter}

According to Raymer in CH 5.3.8 of \cite{Raymer.2006} he states that the time of loiter will be most when the induced drag is three times the parasitic drag. 
$$C_L = 3 C_{D_o}$$

Assuming Loiter to be steady, we compute the wing loading as:-
$$\frac{W}{S} = q \sqrt{3C_{D_o} \pi e AR}$$
For our case, the optimal wing loading is 166.12 $N/m^2$.

Thus, We can infer that we must decrease our loiter speed to a low value to decrease the optimal wing loading in the next iteration. For example, for a loiter velocity of 15 m/s, we will get wing loading to be 130 $N/m^2$.

\subsection{Second weight estimate}

We got a more accurate solutions of values from the previous values. 

From the Power loading we got weight of the battery as well as the propeller and motor configuration. So when we input these values in the formula for the weight estimate, we get:- 

$$W_{0} = \frac{W_{pl}  +  W_{\text{battery}}+  W_{\text{Propeller}} + W_{\text{Motor}}}{1 - \frac{W_{e}}{W_{0}}}$$

Where:-
\begin{itemize}
    \item[-] $W_{\text{battery}}$ is battery weight which is 639 g.
    \item[-] $W_{\text{Propeller}}$ is propeller weight which is 19 g
    \item[-] $W_{\text{Motor}}$ is Motor weight which is 150 g.
    \item[-] $\frac{W_{e}}{W_{0}}$ is the empty weight fraction from initial estimate.
\end{itemize}

\begin{figure}[h]
    \centering
    \includegraphics[width=1.0\linewidth]{Codes//Week 2/weight_2.png}
    \caption{Second weight estimate iteration}
    \label{Second weight estimate iteration}
\end{figure}

According to our new estimate, the weight after tolerance is 6.38 kg.

\afterpage{\clearpage}
\newpage

\textbf{\Huge{Chapter 6}}
\section{Wing Design}
This section estimates the Lift Coefficient for the cruise phase in the UAV mission profile. Various airfoils are evaluated using data from reputable airfoil databases, followed by simulations conducted through the XFLR5 software. We scrutinize performance diagrams and explore multiple wing configurations, factoring in various design parameters such as chord length, span length, high lift devices (e.g., flaps), taper ratio, and sweep angle. Upon achieving performance plots closely aligning with our design specifications and considering practical feasibility, we finalize airfoil and wing configuration for representation.
\vspace{5mm} 
 

\color{red}
\subsection{\large Calculation of Design Lift Coeffient}

      \text{\large \underline{ Cruise}}

\color{black}
The pivotal phase in the mission profile is the Cruise Phase, where the mini UAV operates at an altitude of 100 meters with a velocity of 18 m/s. We aim to select a wing that demonstrates optimal aerodynamic performance during this phase, specifically by maximizing the Lift-to-Drag ratio at the desired operating Lift Coefficient for the cruise. Initially, we calculate the Lift Coefficient in cruise condition \\
Given:\\
 Cruise Velocity: \( v_{\text{cruise}} = 17 \) m/s\\
Density: \( \rho_{atm} = 1.225 \) kg/m³\\
 Takeoff weight: \( W_0 = 62.58 \) N\\
 Reference area: \(S = 0.7 m^2\) 

We can calculate the Lift Coefficient for the cruise using the formula:

\[
C_{L_{\text{cruise}}} = \frac{W_0}{\frac{1}{2} \rho_{atm} v_{\text{cruise}}^2 S}    \tag{6.1} \]

Thus, the lift coefficient for cruise is \( C_{L_{\text{cruise}}} = 0.5048 \). 
\color{red}
\subsection{Calculation of flow parameters}
\color{black}
Estimating flow parameters such as Reynolds number and Mach number for the cruise condition is essential in selecting the most suitable airfoil for our UAV. These parameters provide crucial insights into the aerodynamic behavior of the airfoil at the designated operating conditions, guiding the selection process toward optimal performance and efficiency. 
\color{red}
\begin{itemize}
\item{\underline{Reynolds number}}
\color{black}
\\Reynolds number is a crucial parameter in airfoil selection as it helps determine the aerodynamic behavior of the airfoil under different flow conditions. It is defined as the ratio of inertial forces to viscous forces within a fluid flow regime and is instrumental in predicting flow patterns around an airfoil. In airfoil selection, the Reynolds number provides insights into the transition from laminar to turbulent flow, which significantly impacts the aerodynamic performance and efficiency of the airfoil.

Given the parameters:

 Chord length (\( c \)): 0.312347524 m\\
 Cruise Speed (\( V_{\text{cr}} \)): 17 m/s\\
 Atmospheric Density (\( \rho_{\text{atm}} \)): 1.2256 kg/m³\\
 Atmospheric Viscosity (\( \mu_{\text{atm}} \)) at 15°C: \( 1.81 \times 10^{-5} \) Pa-s

The Reynolds number (\( Re \)) can be calculated using the formula:

\[
Re = \frac{V_{\text{cr}} \cdot c}{\nu_{\text{atm}}} \tag{6.2}
\]

Where \( \nu_{\text{atm}} \) is the kinematic viscosity and is given by \( \frac{\mu_{\text{atm}}}{\rho_{\text{atm}}} \).

Substituting the given values into the formula:

\[
Re = \frac{17 \times 0.312347524}{\frac{1.81 \times 10^{-5}}{1.2256}} \tag{6.3}
\]

\[
Re = 359548.24
\]

Therefore, the Reynolds number for the given parameters is \( 359548.24 \). 
\color{red}
\item{\underline {Cruise Mach Number}}
\color{black}
\\Cruise Mach number is a crucial parameter in airfoil selection as it characterizes the relative speed of the aircraft to the speed of sound in the surrounding air. It is defined as the ratio of the aircraft's velocity to the speed of sound in the medium. Understanding the cruise Mach number is vital in airfoil selection as it helps determine the aerodynamic behavior of the airfoil at high velocities, particularly in transonic and supersonic flight regimes.

Given the parameters:

 Specific heat ratio (\( \gamma \)): 1.4\\
 Air Gas Constant (\( R \)): 287.05287 J/kgK\\
 Atmospheric Temperature at Mean Sea Level (\( T_{\text{atm}} \)): 288.15 K\\
 Cruise velocity (\( V_{\text{cr}} \)): 17 m/s\\

The speed of sound (\( a \)) can be calculated using the formula:

\[
a = \sqrt{\gamma R T_{\text{atm}}} \tag{6.4}
\]
And the cruise Mach number (\( M_{\text{cr}} \)) can be calculated as:

\[
M_{\text{cr}} = \frac{V_{\text{cr}}}{a} \tag{6.5}
\]

Substituting the given values into the formulas:

\[
a = \sqrt{1.4 \times 287.05287 \times 288.15}  = 340.3 \, \text{m/s}
\]

\[
M_{\text{cruise}} = \frac{17}{340.3} = 0.0499 \tag{6.6}
\]

Therefore, the cruise Mach number for the given parameters is  0.0499.
\end{itemize}
\color{black}
\textbf{ \large The parameters below are guiding our airfoil selection.}
\begin{table}[h]
\centering
\begin{tabular}{ll}
\hline
Cruise Speed (m/s) & 17 \\
CL\_cruise & 0.5048 \\
Reynolds Number (Re) & 359,548.24 \\
Cruise Mach Number (M.cruise) & 0.049956804 \\
Maximum Lift-Drag Ratio (L/D)max & 14.56067 \\
\hline
\end{tabular}
\end{table}\\
\color{red}
\subsection{ Airfoil selection}

 \large \underline{SA7035 Airfoil}\\
\color{black}
We have selected the SA7035 airfoil for the above flow parameters, which is suitable for our UAV. The SA 7035 airfoil, developed by the German Aerospace Center (DLR), is highly favored in the UAV and light aircraft sectors for its exceptional aerodynamic properties. Renowned for its high lift-to-drag ratio and reliable stall behavior, it excels in various flight conditions, including cruising and maneuvering.

Featuring a thick profile and optimized camber distribution, the SA 7035 generates ample lift while keeping drag levels low, ensuring efficient performance, especially during extended cruise phases. Its forgiving stall characteristics also enhance safety and maneuverability, appealing to pilots of all skill levels.
\begin{figure}[h]
    \centering
    \includegraphics[width = 0.85\linewidth]{Codes/Week 6/Airfoil.png}
    \caption{SA7035 Airfoil}
    \label{fig:enter-label}
    \end{figure}
\begin{table}[h]
\centering
\caption{SA7035 Airfoil Parameters}
\begin{tabular}{ll}
\hline
Max Cl/Cd & 71.2969 \\
$\alpha_{\text{maxCl/Cd}}$ & 5 deg \\
$C_{\text{Lmax}}$ & 1.2535 \\
$\alpha_{\text{Stall}}$ & 12.25 deg \\
Max thickness & 9.2\% at 27.9\% chord \\
Max camber & 2.4\% at 41.9\% chord \\
\hline
\end{tabular}
\end{table}
\newpage
We will now create graphs illustrating the relationship between Lift Coefficient ($C_L$) and Angle of Attack ($\alpha$), as well as the relationship between Drag Coefficient ($C_D$) and Angle of Attack ($\alpha$). We analyze these graphs to assess whether the airfoil meets the required performance criteria.
\newpage
\begin{figure}[H]
    \centering
     \includegraphics[scale = 0.8]{Codes/Week 6/Cl_alpha.png}
    \caption{Coefficient of lift vs angle of attack}
    \label{fig:enter-label}
\end{figure}

\begin{figure}[H]
    \centering
    \includegraphics[scale = 0.8]{Codes/Week 6/Cd_alpha.png}
    \caption{Coefficient of drag vs angle of attack}
    \label{fig:enter-label}
\end{figure}

\begin{figure}[H]
    \centering
    \includegraphics[scale = 0.8]{Codes/Week 6/Cl_Cd.png}
    \caption{Coefficient of lift vs coefficient of drag}
    \label{fig:enter-label}
\end{figure}

\begin{figure}[H]
    \centering
    \includegraphics[scale = 0.8]{Codes/Week 6/Cl_Cd_ratio.png}
    \caption{Cl/Cd vs angle of attack}
    \label{fig:enter-label}
\end{figure}
\newpage

We have established that the Lift Coefficient ($C_L$) during cruise conditions, which is 0.5048, is achieved at an angle of attack of 2.5 degrees. This value notably falls below the stall angle of attack, suggesting favorable performance attributes of the airfoil. Hence, this airfoil is considered preferable.

\color{red}
\subsection{Wing Configuration}

Low-Wing Configuration:
\color{black}
\\- In the low-wing configuration, the wings are mounted below the aircraft's fuselage.
\color{red}
\\- Advantages:
\color{black}
 \\- Enhanced stability: Low-wing UAVs exhibit better lateral stability during flight, making them well-suited for missions requiring precision control, such as surveillance and mapping.
\\  - Payload capacity: Placing the wings beneath the fuselage allows for larger payload capacity, as payloads can be mounted directly underneath without wing interference.
\\  - Aerodynamic efficiency: Low-wing UAVs may benefit from reduced interference drag between the wings and the fuselage, improving overall aerodynamic efficiency and potentially longer endurance.
\\  - Ground operations: The low-wing configuration facilitates easier ground operations, including launching and landing, especially in confined spaces, as the wings do not obstruct ground clearance.
\color{red}
\\- Disadvantages:
\color{black}
\\  - Vulnerability to ground debris: With the wings positioned beneath the fuselage, low-wing UAVs are more susceptible to damage from ground debris during takeoff and landing, which could potentially lead to damage to the wings or payload.
 \\ - Limited ground clearance: Low-wing UAVs may have limited ground clearance, which could pose challenges when operating on rough terrain or uneven surfaces.
\\  - Visibility: While low-wing configurations offer good visibility for the payload, certain types of sensors or cameras mounted on top of the fuselage may experience slightly reduced visibility.

\color{red}
Mid-Wing Configuration:
\color{black}
\\- In the mid-wing configuration, the wings are mounted at the midpoint of the fuselage.
\color{red}
\\- Advantages:
\color{black}
\\  - Balanced lift distribution: Mid-wing UAVs typically achieve a balanced lift distribution, enhancing stability and control during flight maneuvers.
 \\ - Aerodynamic efficiency: Similar to low-wing configurations, mid-wing UAVs can benefit from reduced interference drag between the wings and the fuselage, contributing to overall aerodynamic efficiency.
\\  - Visibility: Mid-wing designs provide good visibility for sensors and cameras mounted on the fuselage, allowing for effective surveillance and reconnaissance missions.
\\  - Payload flexibility: The mid-wing configuration allows for flexible payload integration options, as payloads can be mounted on top of the fuselage without interference from the wings.
 \\ - Ground clearance: Mid-wing UAVs typically have sufficient ground clearance for landing gear and other components, making them suitable for various terrain conditions.
\color{red}
\\- Disadvantages:
\color{black}
\\  - Complexity: Mid-wing configurations may involve more complex structural design and integration, especially when considering payload mounting and aerodynamic considerations.
\\  - Maintenance accessibility: Accessing components on top of the fuselage, such as sensors or cameras, may require additional effort and time compared to configurations with lower wing wings.
\\  - Vulnerability to damage: The mid-wing position exposes the wings to potential damage during ground operations, such as takeoff and landing, especially in rough terrain.
 \\ - Weight distribution: Achieving optimal weight distribution in mid-wing UAVs can be challenging, as the payload and other components need to be carefully balanced to maintain stability and performance.

\color{red}
High-Wing Configuration:
\color{black}
\\- In the high-wing configuration, the wings are mounted above the aircraft's fuselage.
\color{red}
\\- Advantages:
\color{black}
\\  - Excellent visibility: High-wing UAVs provide unobstructed visibility for sensors, cameras, and other payloads mounted beneath the fuselage, facilitating effective surveillance and reconnaissance missions.
\\  - Stability: High-wing configurations typically offer excellent inherent stability, especially during banking maneuvers, making them suitable for various applications, including aerial mapping and monitoring.
\\  - Protection from ground debris: With the wings positioned above the fuselage, high-wing UAVs are less susceptible to damage from ground debris during takeoff and landing, enhancing durability and reliability. Our payload module will also be safe.
\\  - Payload flexibility: High-wing designs allow flexible payload integration options, as payloads can be mounted beneath the fuselage without wing interference.
 \\ - Ease of ground operations: High-wing UAVs often feature ample ground clearance, making takeoff and landing operations easier, especially in rough or uneven terrain.
\color{red}
\\- Disadvantages:
\color{black}
\\  - Aerodynamic interference: High-wing configurations may experience increased interference drag between the wings and the fuselage, potentially impacting overall aerodynamic efficiency and endurance.
\\  - Limited maneuverability: High-wing UAVs offer stability but may have slightly reduced maneuverability compared to other configurations, which can be considered for specific mission profiles.
\\  - Weight distribution: Achieving optimal weight distribution in high-wing UAVs can be challenging, as the payload and other components must be carefully balanced to maintain stability and performance.
\\  - Complexity in payload integration: Mounting specific payloads, such as gimbals or sensors, beneath the fuselage of high-wing UAVs may require more complex integration and mounting solutions compared to configurations with the wings positioned lower.
\vspace{5mm} 
\\By considering all the wing configurations, we have decided to go with\textbf{ high wing configuration} for our UAV.
\begin{figure}[H]
    \centering
    \includegraphics[width=\linewidth]{Codes/Week 6/wing configuration.jpeg}
    \caption{High Wing configuration}
    \label{fig:enter-label}
\end{figure}
\color{red}
\subsection{Wing Design}
\color{black}
The following are the detailed wing parameters that were previously estimated.
\begin{table}[H]
\centering
\begin{tabular}{l l}
\textbf{Parameter} & \textbf{Value} \\
\hline
Reference Area (S) & 0.7 sq.m \\
Wing Geometry & Rectangular Wing \\
Aspect Ratio (AR) & 7.175 \\
Wingspan (b) & 2.241093483 m \\
Chord (c) & 0.312347524 m \\
\end{tabular}
\caption{Wing Parameters}
\label{tab:wing_parameters}
\end{table}
\begin{figure}[H]
    \centering
    \includegraphics[width=0.8\linewidth]{3d_wing.png}
    \caption{3D MODEL OF RECTANGULAR WING}
    \label{fig:enter-label}
    \end{figure}
\color{red}
 \large{\underline{Taper ratio}}
\color{black}
\vspace{5mm}
\\Taper ratio refers to the ratio of the wingtip chord to the wing root chord, providing insight into the shape of the wing planform. A taper ratio of 1 indicates a rectangular wing, where the chord length remains constant from the wing root to the wingtip. This configuration is contrasted with tapered wings, where the chord decreases progressively from the root to the tip.

Rectangular wings offer several advantages over tapered wings. Firstly, they provide more straightforward structural design and analysis due to their uniform geometry, facilitating ease of fabrication and reducing manufacturing complexities. Additionally, rectangular wings typically exhibit more predictable aerodynamic characteristics, especially at low speeds, simplifying flight performance analysis and enhancing overall stability. Moreover, the uniform chord distribution of rectangular wings often results in improved stall behavior and handling characteristics, making them preferable for specific applications, such as UAVs operating in varied flight conditions.

 Given these considerations, the decision to adopt a rectangular wing configuration in which the\textbf{ Taper ratio = 1 }is taken.\\
 \vspace{5mm}
\color{red}

\large{\underline{Wing Setting Angle $i_w$
}}\\
\color{black}
The wing setting angle, denoted as \( i_w \), refers to the angle at which the wing is positioned relative to the aircraft's fuselage or horizontal reference plane. It plays a crucial role in determining an aircraft's aerodynamic performance and stability characteristics during flight. 

A positive wing setting angle typically results in a nose-up attitude, generating more lift but potentially increasing drag. Conversely, a negative setting angle may reduce lift while enhancing aerodynamic efficiency. The optimal setting angle depends on various factors, including aircraft design, mission requirements, and flight conditions.

In our wing design, as the desired lift coefficient has been achieved,  we are maintaining the wing setting angle at 0 degrees. This indicates that the wing is positioned parallel to the horizontal reference plane. This configuration also suggests an optimal balance between lift generation and drag minimization, aligning with the desired aerodynamic performance criteria for the aircraft's mission profile.\\
\vspace{5mm}
\color{red}
\large{\underline{Wing Sweep and Geometric Twist}}\\
\color{black}
Wing sweep refers to the angle at which the wings are inclined backward from the root to the tip. This feature can enhance aerodynamic efficiency by reducing drag at high speeds, improving stability, and delaying the onset of compressibility effects. Geometric twist, on the other hand, involves varying the angle of incidence along the span of the wing, typically with a higher angle of incidence at the wing root and decreasing towards the wingtip. Geometric twist helps to optimize lift distribution and stall characteristics across the wing span.

In our wing design, wing sweep and geometric twist are not utilized due to concerns related to fabrication complexity. Incorporating these features would require intricate structural design and manufacturing processes, potentially increasing production costs and time. Additionally, the introduction of wing sweep and twist could introduce structural challenges and compromises in aerodynamic performance.

Furthermore, the decision not to employ winglets, which are small wing-like structures often added to the wingtips, is made to simplify the design and manufacturing process further. While winglets can reduce induced drag and improve fuel efficiency, their integration adds complexity to the wing design and may not provide significant benefits for the intended mission profile.\\
\vspace{5mm}
\color{red}
\large{\underline{Ailerons}}\\
\color{black}
Ailerons are control surfaces typically located on the trailing edge of the wing near the wingtips. They are responsible for controlling the aircraft's roll motion by deflecting in opposite directions. When one aileron moves upward, the other moves downward, causing the aircraft to roll about its longitudinal axis.

In our wing design, the placement of the ailerons is crucial for optimizing control effectiveness and aerodynamic performance. Therefore, the ailerons will be positioned at specific locations along the wing. Specifically, they will be located between \textbf{0.5 to 0.96 of the wing's span}, ensuring sufficient leverage for roll control across the entire wing length. Additionally, the ailerons will span from \textbf{0.8 to 1 of the wing's chord}, providing adequate surface area for effective control authority while maintaining aerodynamic efficiency.

\vfill

\clearpage

\newpage
\textbf{\Huge{Appendix}}
\appendix


\section{References}

\bibliographystyle{IEEEtran}
\bibliography{References}

\vspace{10 pt}

The GitHub account having all the codes can be accessed as: - 

\href{https://github.com/abhijeetmangela/Group_7_design.git}{\text{https://github.com/abhijeetmangela/Group\textunderscore 7\textunderscore design.git}}

\newpage

\section{Changes}

\subsection{Week 2}
\begin{enumerate}
    \item The Data collection part was completely changed.
    \item The general design of the pdf was changed.
    \item More data was added along with a table for better comparison.
    \item Battery performance was estimated.
    \item Weight was estimated.
    \item References were added.
    \item The Mission profile was updated.
    \item Details on mission profile were added.
\end{enumerate}

\subsection{Week 3}
\begin{enumerate}
    \item The Weight estimation was redone
    \item Power was estimated for different phases 
\end{enumerate}

\subsection{Week 4}
\begin{enumerate}
    \item Power calculations were recalculated.
    \item Battery was selected.
    \item Motor and propeller were selected.
    \item Wing loading was done
\end{enumerate}

\newpage


\section{Contributions}

\subsection{Week 2}

\subsubsection{Abhijeet Mangela AE21B040}
Wrote full Latek report, Drew the mission profile with Inkscape and Autocad, Empty weight Fraction estimation, and Final weight estimation with Senthil

\subsubsection{Navin Yadav AE23M803}

Data Collection; Literature Survey

\subsubsection{Balamurugan S AE23M009}

Data Collection; Literature Survey

\subsubsection{Samarth R Krishna AE23M032}

Data Collection; Literature Survey

\subsubsection{Senthil B AE23M035}

Detailed Mission Profile; Preliminary Weight Estimation (only iteration); Battery weight estimation.

\subsubsection{Rajendran Anandhu Nair AE23M027}

Data Collection; Literature Survey




\subsection{Week 3}


\subsubsection{Abhijeet Mangela AE21B040}
Changed the Mission Profile in Latex, Converted references to BibTeX, Did some cleanup of document, did the Preliminary weight estimate, which was wrong last week, Power estimation for cruise and climb, Linked all files internally in Github for easy connectivity.

\subsubsection{Navin Yadav AE23M803}
The analytical calculation and added these calculations on latex 

\subsubsection{Balamurugan S AE23M009}
Data Collection; Literature Survey; Max L/D vs Sqrt(Wetted AR) Calculations and Plotting.


\subsubsection{Samarth R Krishna AE23M032}
The analytical calculation (Drag Polar calculation, L/D max estimation, calculation of Thrust and Power required at different phases), and added these calculations on latex


\subsubsection{Senthil B AE23M035}
Data Collection; Image Processing using ImageJ; Max L/D vs Sqrt(Wetted AR) Calculations and Plotting.


\subsubsection{Rajendran Anandhu Nair AE23M027}
Data Collection; Image Processing using Fusion 360; Report writing for Reference and Wetted areas, Aspect ratios, Coefficient of lift, and Maximum lift-to-drag ratio. 

\subsection{Week 4}


\subsubsection{Abhijeet Mangela AE21B040}
Did Wing loading for stall criteria, effect of flaps, as well as all $\frac{W}{S}$ calculations for Stall, Takeoff, Landing, Cruise and loiter and wrote \LaTeX for it.

\subsubsection{Navin Yadav AE23M803}
The analytical calculation and added these calculations on latex 

\subsubsection{Balamurugan S AE23M009}


\subsubsection{Samarth R Krishna AE23M032}
The analytical calculation (Iterations and calculation of Power required at different phases), and added these calculations on latex


\subsubsection{Senthil B AE23M035}
Calculated power loading calculations for cruise climb descent,loitre and stall conditions. Also calculated and approximated some design parameters like wing reference area, maximum Lift to Drag Ratio using previously collected data.

\subsubsection{Rajendran Anandhu Nair AE23M027}
Calculated battery capacity required for each mission segment based on their power requirements. Selected battery, motor and propeller required for the UAV. Wrote the power loading, battery capacity and powerplant selection part in LATEX. 


\newpage

\section{Current estimates}
\begin{itemize}
    \item Range- 20 Km
    \item Endurance- 30 min
    \item Payload- 1.4 kg
    \item Total weight- 6.5 Kg
\end{itemize}

\end{document}
